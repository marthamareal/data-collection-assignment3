\documentclass[a4paper,11pt]{article}
\usepackage{graphicx}
\usepackage{url}
\usepackage{biblatex}
\bibliography{referconcp3.bib}
\graphicspath{{images/}}
\begin{document}

\begin{titlepage}


\title{Traffic Data Collection }
\author{NAGGINDA MARTHA   15/U/9170/PS 215013711}
\date{\today}
\end{titlepage}
\cleardoublepage
\maketitle
\tableofcontents
\cleardoublepage
\section{Abstract}
\paragraph{
The objective of this project was to i dentify araes with roads which are highly affected by traffic and improve on the effiecence in in movement. The project involved People who use both public and private means. This was done through providing forms to these people and record the traffic status on differnt roads at different time intervals. The forms are then saved on the cloud server. 
}
\section{Introduction}
\paragraph{Traffic jam refers to a situation where a large number of vehicles are close to each other uneble to move or moving slowly.In most cases trffic jam occurs because the roads are blocked by something like accidents, heavy rains and other factors.}
\subsection{Background}
\paragraph{In many countries , there is an increased rate of traffic jam due to increase number of vehicles. For example in Uganda according to Uganda Revenue Authority(URA), there is an estimate increase of vehicles by over 635,656 from 50,102 in the last 20 years. 80 percent of these vehicles operate in urban areas there by leading to traffic jam. }
\subsection{Problem Stetment}
\paragraph{Currently most drivers have no information about traffic  on different roads at a given time. therefore they end up using roads which are so much congested and tend to spend much time on their travel.}
\paragraph{Collection of this data will help people to know how they can avoid traffic by looking at the roads in the sample which are highly affected by traffic.}
\cleardoublepage
\section{Methods}
\paragraph{This research was conducted using a form which was built using ODK build and an application called ODK collect which users were using to fill the for.the forms filled were then sent to the Aggregate server. Above are the screenshots of the mobile application.
 }

 \begin{figure}[h]
\includegraphics[width=3cm,height=4cm,scale=1.2]{sc1}
\includegraphics[width=3cm,height=4cm,scale=1.2]{sc2}
\includegraphics[width=3cm,height=4cm,scale=1.2]{sc3}
\includegraphics[width=3cm,height=4cm,scale=1.2]{sc4}
\includegraphics[width=3cm,height=4cm,scale=1.2]{sc5}
\includegraphics[width=3cm,height=4cm,scale=1.2]{sc6}
\includegraphics[width=3cm,height=4cm,scale=1.2]{sc8}
\includegraphics[width=3cm,height=4cm,scale=1.2]{sc7}
\end{figure}
\cleardoublepage
\section{Results}
\begin{table}[h]
\caption{Traffic Jam Results Table}
\begin{tabular}{|l|l|l|l|}
\hline
COUNTRY &DISTRICT & ROAD& DATE AND TIME\\ \hline
Uganda   & Kampala  & Kampala Road & 14/05/2017 9:45pm \\ \hline
Uganda   & Jinja  & Jinja Road   & 18/05/2017 3:32pm \\ \hline
Uganda  & Gulu & Kamdini Road  & 17/05/2017 8:55am \\ \hline
Uganda  & Mbale & Bumbobi Road  & 16/05/2017 7:42am \\ \hline
Uganda  & Mbarara & Kikagati Road  & 17/05/2017 9:12am \\ \hline
\end{tabular}
\end{table}
\paragraph{The table above shows the percentage of how roads are affected by traffic jam at different time intervals}
\paragraph{Above is the pie chart showing the results from the reasearch}
\begin{figure}[h]
\includegraphics[width=12cm,height=8cm,scale=1.2]{visual}
\end{figure}
\cleardoublepage


\section{Conclusion}
\paragraph{Accrording to the analysis on the data collected during the research,It shows that traffic jam affects people in urban centers mostly and also roads which are poorly constructed in that they are easily blocked in case of heavy rain.This shows that there is need for road development in urban centers.Meanwhile the people can use the available data to determine which roads to use at aparticular time and avoid traffic jam}
\end{document}